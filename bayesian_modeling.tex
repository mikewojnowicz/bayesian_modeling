
 \pdfoutput=1

\documentclass{article} % For LaTeX2e

%STANDARD PREAMBLE
%https://tex.stackexchange.com/questions/68821/is-it-possible-to-create-a-latex-preamble-header
\usepackage{/Users/mwojno01/Research/Learning/latex_preamble/preamble}

\begin{document}

\title{Bayesian Modeling} 

\maketitle
\tableofcontents

\section{Overview}

\subsection{Goal}  The goal of this workshop is to introduce students to the concepts and practice of Bayesian modeling.  

\subsection{Target Audience}  We expect that the typical student will be a graduate student, faculty member, staff member, or researcher in a quantitative field (such as computer science,  statistics,  engineering, or biology),  who would like to learn more about Bayesian modeling.  

\subsection{Prerequisites} Prerequisites include calculus,  some linear algebra, and some familiarity with introductory probability (e.g.,  we will assume prior familiarity with concepts such as expectation,  conditional probability,  and commonly used distributions,  such as Gaussian and Poisson.)      We will work in Python when working collaboratively as a class,  but you can do offline analyses in any language of your choosing. 

\subsection{Format}

We will present lectures via slides.   But I would like there to be student contribution,  active learning (in a non-artificial sense),  and some student autonomy and self-direction. 

To this end,  I am leaning leaving the last chunk of the workshop  open for students to present a mini-project they did on Bayesian data analysis.  Probably this would be chosen after covering the material in Section \ref{sec:more_complicated_models}.

That said,  along the way,  we could also potentially try to encourage student involvement via discussions and active-learning components embedded within the material,  e.g.

\begin{itemize}

\item Student ``lightning chat" (10 minute) presentations?  Could do one per student per workshop.  These could consist of students choosing any of the following:
	\begin{itemize}
	\item Presentation of Python implementations of models from \cite{hoff2009first} ,  \cite{ gelman2013bayesian},  or the workshop.
	\item Presentation of an exercise from \cite{gelman2013bayesian}.
	\item Presentation of a reading section,  blog,  etc.  of interest.
	\item Presentation of a mathematical derivation of something relevant to the course.
	\item Presentation of mid-progress on their student project.
	\end{itemize}
\item Real-time python applications lab -- Google Collab exercises ?  Python (rather than R) implementations of \cite{hoff2009first} and \cite{gelman2013bayesian} ?  Progress on their project?  Etc.
\item Mini reading group discussions 
\end{itemize}


\section{Topics}

Below are topics we plan to cover in the course:

\subsection{Introduction to Bayes}

We present everything in here using conjugate models with closed-form posteriors.  The models are useful in and of themselves,  as well as to build intuition for more complicated models.   

Primary references here are \cite{hoff2009first} and \cite{gelman2013bayesian}.

\begin{itemize}
\item \textbf{Why Bayes?} -- See Section 1.3 of \cite{hoff2009first}.   \cite{bishop2006pattern} has some nice plots motivating why use Bayesian linear regression over standard linear regression.   \cite{ghahramani2013bayesian}  has some nice plots illustrating the Bayesian approach and how it mitigates overfitting.   I can provide a nice example with biometric profiling of human typing dynamics.   \cite{held2006bayesian} has a nice simple example of obtaining non-standard functionals from the posterior that can be of interest.   \cite{wilson2020case} presents the case for Bayesian deep learning.  
\item \textbf{Belief functions,  Bayes rule} -- Sections 2.1,  2.2 of \cite{hoff2009first}.   \cite{ghahramani2013bayesian} briefly overviews of the Bayesian framework.  \textit{Why most published research findings are false} \cite{ioannidis2005most} provides nice motivation.    Could perhaps cover exchangeability here. 
\item \textbf{Binomial, Poisson, normal, multivariate normal models} -- Sections 3.1,  3.2, 5, and 5 of \cite{hoff2009first}.     Introduce the exponential family formalism \cite{wojnowiczXXXXexponential} for much greater breadth. 
\item \textbf{Bayesian linear regression} -- Section 9 of \cite{hoff2009first}.     I have notes on this.   There are some nice slides here which also illustrate the use of kernels.\footnote{Nice Bayesian linear regression slides: \url{https://www.cs.toronto.edu/~rgrosse/courses/csc411_f18/slides/lec19-slides.pdf}}   Introduce model selection here (Section 9.3 of \cite{hoff2009first}. 
\end{itemize}

We will want to find a way to get students to group up,  probably based on domain expertise/interests,  so that they can eventually work together on a project.  

\subsection{Methods}

We introduce these methods,  which can be used for models without closed-form posteriors.  We practice applying them in the next section.  

\begin{itemize}
\item \textbf{MCMC} - \blue{Karin} will present.
\item \textbf{Variational inference} \cite{wojnowiczXXXXfoundations}.
\end{itemize} 

\subsection{More complicated models} \label{sec:more_complicated_models}

Here are some models which are still fairly standard,  but lack conjugate priors,  and so inference typically requires VI or MCMC.     \blue{Karin: Where here,  or elsewhere,  would you like to illustrate applications of MCMC?}

\begin{itemize}
\item \textbf{Hierarchical models}  Hierarchical normal model (e.g. Gelman's 5 schools example),  hierarchical linear regression (Chapter 13 of \cite{gelman2013bayesian}).
\item \textbf{Regression models for binary and multi-class data} Includes logistic regression, probit regression,  binomial,  multinomial,  etc.    Use this to cover additional inference techniques:  auxilliary variable trick and Laplace variational inference.      See also pp.  390 of \cite{hoff2009first} for a useful warm-starting strategy.   Could generalize to Bayesian GLMs.   Could also cover or mention hierarchical extensions (i.e.  Bayesian GLMM's). 
\item \textbf{Mixture models} I will give CAVI for Gaussian mixture models. 
\item \textbf{Time series models}  Probably just hidden markov models,  although would be nice to also introduce state space models.    Could mention embedding of GLM's or GLMM's within them.   May give some overview to probabilistic graphical models here.  
\end{itemize}

\subsection{Even more complicated models} 

Here we discuss models which have additional complexity -- they could involve neural networks,  non-parametrics,  larger scale,  etc.    Often these involve some additional inferential machinery -- \textit{stochastic} variational inference,  reparametrization trick,  etc.

%The likelihood that students would use this material for their projects is relatively low,  so this would make a nice set of things to cover as we begin transitioning over to the student project day.

\begin{itemize}
\item \textbf{Why Bayesian Deep Learning?}  Bayes and neural networks.   20-30 min w/ guest presenter,  Kyle Heuton,  Ph.D.  student,  computer science.
\item \textbf{Custom models}  Could present black-box variational inference or automatic differentiation variational inference here.   I have some notes that I could convert to slides.  Would take some work though.  \blue{Anna,  perhaps you'd be interested,  since you've been working with this?}
\item \textbf{Sampler platter of other possible topics}   VAE,   Gaussian processes,  composing time series models with neural networks,  indian buffet processes,  dirichlet process mixture models,  etc.  
\end{itemize}

\subsection{Bayesian workflow}  \label{sec:Bayesian workflow}


Lots of nice resources for Bayesian workflow.    For example: \cite{gelman2020bayesian} or \cite{gabry2019visualization}.  Section 6 of \cite{gelman2013bayesian} covers model checking.       

We could cover this the day before the student projects -- which should help them as they wrap up their projects -- and leave lots of time/space for workshopping. 

\subsection{Student project presentations}

For student projects,  we could have students present results of a Bayesian data analysis mini-project.   Perhaps they could highlight some aspect of the Bayesian workflow along the way. 

We would probably want to have a ``workshopping" section the day before they do their presentations.


\bibliography{references_bayesian_modeling}{}
\bibliographystyle{unsrt}


\appendix

\section{Resources which may be appropriate for mini-reading group or student presentations}

\begin{itemize}
\item Intro: \textit{Why most published research findings are false. } \cite{ioannidis2005most}
\item Model checking: Bayesian workflow.
\item Textbook sections TBD. 
\end{itemize}

\section{Leads on concrete exercises / projects}

\subsection{Batting average dataset}

The hierarchical normal model for (arcsine-transformed) batting average data on pp.  163 of \cite{gelman2013bayesian} has some serious deficiencies,  as exposed in Table 6.1 in the section on model checking.    

Can you construct (and learn) a better model which makes predictions closer to the true final batting average?
 
Examples:
\begin{itemize}
\item Add an extra layer to the hierarchy,  so that player $p$'s 1970 batting average inherits from player $p$'s overall batting average which in turn inherits from a population batting average.  (Of course,  I am speaking of the arcsine-transformed batting averages,  so that we can use a hierarchical normal model.)
\item Add an autoregressive component, because,  as mentioned by Gelman,  player batting averages \textit{DO} change over time. 
\end{itemize}

The text also does a poor job of checking the modeling assumption violations that were of concern.   Can you do a better job of checking them,  and if necessary,  address them?

Examples:
\begin{itemize}
\item If batting averages are indeed heavy tailed or skewed,  move from a normal distribution to something else.   For example,  could try a t-distribution with Laplace inference to handle the non-conjugacy. 
\item If the variance is indeed too high for a binomial model,  try something that can handle the overdispersion. 
\end{itemize}


\section{Need to do}

\begin{itemize}
\item \blue{Karin}  Bring pymc3,  stan,  etc.  into this -- will make things a lot more useful for the audience than requiring that they code things up from scratch!
\end{itemize}

%\subsection{Regression models for binary and multi-class data}
%
%Take the batting average problem on pp.  163 of \cite{gelman2013bayesian} and use a more conventional modeling approach (logit or probit regression) than what was described in the text (using the arcsine transformation).   What kinds of inference results do you get?  How far off is  Polya-Gamma variational inference?  What about Laplace variational inference? 


          


\end{document}
